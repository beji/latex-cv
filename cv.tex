\documentclass[10pt,ngerman,a4paper]{article}
\usepackage[margin=3cm]{geometry}
\usepackage[utf8]{inputenc} % this is needed for umlauts
\usepackage[ngerman]{babel} % this is needed for umlauts
\usepackage[T1]{fontenc}    % this is needed for correct output of umlauts in pdf
\title{Lebenslauf}
\author{Björn Erlwein}
\date{\vspace{-5ex}} % remove the date

% Do not indent paragraphs
\setlength{\parindent}{0in}
% leave halve a line after each paragraph
\setlength{\parskip}{0.5em}

% Enable multicolumns
\usepackage{multicol}
\setlength{\columnsep}{-3.5cm}

% TYPOGRAPHY
%--------------------------------
\usepackage{fontspec} 
\usepackage{xunicode}
\usepackage{xltxtra}
%\setmainfont{Cambria}
\setmainfont{Calibri}

% for images
\usepackage{graphicx}
\graphicspath{ {./} }

% disable numbers before section headings
\setcounter{secnumdepth}{0}

\begin{document}
% disable page numbers
\pagenumbering{gobble}

\maketitle

\section{Persönliche Daten}
\begin{minipage}{0.3\textwidth}
\begin{tabular}{ll}
\textbf{Geburtsdatum} & 25.08.1988\\
\textbf{Geburtsort} & Bayreuth\\
\textbf{Staatsangehörigkeit} & Deutsch\\
\textbf{Sprachen} & Deutsch, Englisch
\end{tabular}
\end{minipage}
\begin{minipage}{0.6\textwidth}
\flushright
\includegraphics[scale=0.5]{portrait}
\end{minipage}
\subsection{Kontakt}

Sauerbruchstraße 33\\
95447 Bayreuth

\begin{tabular}{ll}
\textbf{Telefon} & 0176/56121312\\
\textbf{E-Mail} & bjoernerlwein@gmail.com
\end{tabular}

\section{Schulischer Werdegang}

\begin{tabular}{ll}
\textbf{2000 - 2018} & Gymnasium Christian Ernestinium, Bayreuth
\end{tabular}

\section{Beruflicher Werdegang}

\begin{tabular}{lll}
\textbf{2018 - aktuell} & Fullstack Software Developer & empiriecom GmbH \& Co. KG \\
 & Technologien/Sprachen & Node.js, Typescript, React, webpack, Gitlab,\\
 & & Gitlab CI, terraform, Kubernetes, Istio, Google Cloud \\
\\
\textbf{2015 - 2018} & Frontend Software Developer & empiriecom GmbH \& Co. KG \\
 & Technologien/Sprachen & JavaScript, jQuery, mustache, LESS, Java \\
\textbf{2012 - 2015} & Frontend Software Developer & BAUR Versand GmbH \& Co. KG\\
 & Technologien/Sprachen & JavaScript, jQuery, Java \\
\textbf{2009 - 2012} & Ausbildung Informatikkaufmann & BAUR Versand GmbH \& Co. KG \\
 & Technologien/Sprachen & Java, JavaScript, jQuery \\
\end{tabular}

\section{Schwerpunkte}
\begin{itemize}
\item Konzeption und Entwicklung von microfrontend- und microservicebasierten Cloudsystemen
\item Continuous Integration \& Continuous Delivery
\item Test Driven Development
\item Infrastructure as code
\item Node.js Entwicklung auf Basis von Typescript \& React
\end{itemize}

% \newpage


\section{Vergangene Großprojekte innerhalb der empiriecom GmbH \& Co. KG}
\subsection{Entwicklung einer microfrontend basierten Shopplattform}
Konzeption und Grundlagenentwicklung einer Softmigration des monolithischen Shopsystems auf eine Microfrontend basierte Plattform; Einführung CI/CD \& TDD; Entwicklung einer Cloudbasierten Infrastruktur mit Google Cloud, Kubernetes und Istio.

\begin{tabular}{ll}
\textbf{Rolle} & Technische Projektleitung, Architekt, Fullstack Entwickler\\
\textbf{Zeitraum} & Februar 2018 - Februar 2019\\
\textbf{Entwicklungsmethode} & Kanban\\
\textbf{Technologien/Sprachen} & Node.js, Typescript, React, webpack, Gitlab,\\
& Gitlab CI, terraform, Kubernetes, Istio, Google Cloud
\end{tabular}

\subsubsection{Aufgabenbereich / technische Beschreibung}
\begin{itemize}
\item Entwicklung eines Proof of Concept für eine schrittweise Migrationsstrategie auf Basis von Microfrontends
\item Nahtlose Integration zwischen Alt- und Neusystem die einen schrittweisen Wechsel von einzelnen Seitenbereichen erlaubt
\item Einführung von CI/CD basierter Entwicklung
\item Infrastructure as Code mit terraform
\item Multicluster Servicemesh auf Basis von Istio
\item Entwicklung von Softwarebibliotheken und -vorlagen für Microfrontend Fragmente basierend auf
\begin{itemize}
\item Node.js
\item Typescript
\item React
\item Webpack
\end{itemize}
\item Entwicklung von Deployment- \& Routingstrategien die einfaches und schnelles Traffic Shifting ermöglichen
\item Multimandantenfähige Deploymentpipelines und Fragmente
\end{itemize}

\subsection{Migration der Teams auf die microfrontend basierten Shopplattform}
Entwicklung von Microfrontend Fragmenten in den Teams; Schrittweiser Rollout 
der neuen Fragmente

\begin{tabular}{ll}
\textbf{Rolle} & Technische Projektleitung, Architekt, Support \\
\textbf{Zeitraum} & Februar 2019 - September 2019\\
\textbf{Entwicklungsmethode} & Kanban/Scrum\\
\textbf{Technologien/Sprachen} & Node.js, Typescript, React, webpack, Gitlab,\\
& Gitlab CI, terraform, Kubernetes, Istio, Google Cloud
\end{tabular}

\subsubsection{Aufgabenbereich / technische Beschreibung}
\begin{itemize}
\item Schulung der Teams auf die technische und fachliche Basis von EMMA
\item Fachlicher und Technischer Support der Teams bei der Entwicklung der Fragmente
\item Wartung, Betrieb und Weiterentwicklung der Kernsysteme
\end{itemize}

\subsection{Relaunch Mein Konto}
Neuentwicklung des Mein Konto Bereichs auf Basis der neuen Design Library

\begin{tabular}{ll}
\textbf{Rolle} & Fullstack Developer \\
\textbf{Zeitraum} & Februar 2020 - aktuell\\
\textbf{Entwicklungsmethode} & Kanban/Scrum\\
\textbf{Technologien/Sprachen} & Node.js, Typescript, React, Gitlab, Gitlab CI, Google Cloud
\end{tabular}

\subsubsection{Aufgabenbereich / technische Beschreibung}
\begin{itemize}
\item Layout Umsetzungen mit React
\item Aufsetzen und Integration der Infrastruktur in die EMMA Architektur
\item Auslagern bestehender Logiken in neue Services
\end{itemize}

\subsection{Clientseitiges Rendering der Produklisten und Suchergebnisseiten}
Wechsel eines rein serverseitigen Renderns zu clientseitigem Rendering

\begin{tabular}{ll}
\textbf{Rolle} & Konzeption und Entwicklung \\
\textbf{Zeitraum} & Dezember 2017 - Februar 2018\\
\textbf{Entwicklungsmethode} & Scrum\\
\textbf{Technologien/Sprachen} & Javascript, jQuery\\
\end{tabular}

\subsubsection{Aufgabenbereich / technische Beschreibung}
\begin{itemize}
\item Rein clientseitige Renderlogik für
\begin{itemize}
\item Filter
\item Paging
\item Sortierung
\end{itemize}
\item Caching Vereinfachung für serverseitiges Rendering
\end{itemize}

\subsection{Serviceworker basierter Cache}
Integration eines Serviceworker basierten caches für HTML und statische Dateien

\begin{tabular}{ll}
\textbf{Rolle} & Konzeption und Entwicklung \\
\textbf{Zeitraum} & Oktober 2017 - November 2017\\
\textbf{Entwicklungsmethode} & Scrum\\
\textbf{Technologien/Sprachen} & Javascript\\
\end{tabular}

\subsubsection{Aufgabenbereich / technische Beschreibung}
\begin{itemize}
\item Integration des serviceworkers
\item Optimierung des generierten HTML Codes um cachebare Bereiche zu identifizieren
\item Implementierung von Austauschlogiken für nicht cachebare Shopbereiche
\end{itemize}

\subsection{Evaluierung Suchtechnologie}
Analyse mehrerer Suchsysteme im Vergleich zur verwendeten Lösung

\begin{tabular}{ll}
\textbf{Rolle} & Analyse \\
\textbf{Zeitraum} & Juli 2017 - November 2017\\
\textbf{Entwicklungsmethode} & Scrum
\end{tabular}

\subsubsection{Aufgabenbereich / technische Beschreibung}
\begin{itemize}
\item Erstellung einer Anforderungsmatrix anhand des bestehenden Systems
\item Analyse mehrerer Suchsysteme anhand der Anforderungsmatrix
\item Entwicklung einer Entscheidungsgrundlage für Geschäftsführung
\end{itemize}

\subsection{Migration eines Mandantenshops auf neue Plattform}
Migration auf die neu eingeführte Shopplattform

\begin{tabular}{ll}
\textbf{Rolle} & Teilprojektleitung Entwicklung \& Frontend Developer \\
\textbf{Zeitraum} & November 2016 - März 2017\\
\textbf{Entwicklungsmethode} & Wasserfall\\
\textbf{Technologien/Sprachen} & Javascript, jQuery, LESS, Java\\
\end{tabular}

\subsubsection{Aufgabenbereich / technische Beschreibung}
\begin{itemize}
\item Umsetzung neuer Designvorgaben
\item Migration der Logiken und Contentseiten
\end{itemize}

\section{Ehrenamtliche Projekte}
\subsection{Helping Hands}
Entwicklung einer Plattform für Einkaufs- und Nachbarschaftshilfe in Zeiten von Corona.

\begin{tabular}{ll}
\textbf{Rolle} & Backend Developer \\
\textbf{Zeitraum} & März 2020 - aktuell\\
\textbf{Entwicklungsmethode} & Scrum\\
\textbf{Technologien/Sprachen} & Node.js, Typescript, Firebase, Google Cloud\\
\end{tabular}

\subsubsection{Aufgabenbereich / technische Beschreibung}
\begin{itemize}
\item Modellierung von Datenfluss und Zustand der Daten
\item Entwicklung eines Node.js basierten API-Service für eine React basierte App
\item Hosting und Betrieb von API-Service und Webseite
\item Feature Konzeption
\end{itemize}


% END
\end{document}